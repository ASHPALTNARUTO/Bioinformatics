\documentclass{article}
\usepackage[utf8]{inputenc}

\title{CSE 5370: Bioinformatics}

\author{Venkat Ankit Gundala}

\begin{document}


\section{Substitution Matrix}
Given that, A ←→ G and T ←→ C) are less common
than tranversions (A ←→ T , A ←→ C, G ←→ T , and G ←→ C)
\newline The substitution matrix for above changes is
\newline[[0 A G T C]
\newline[A 1  -2 -1 -1]
\newline[G -2 1 -1 -1]
\newline[T -1 -1 1 -2]
\newline[C -1 -1 -2 1]]

 


\section{Global Alignment}
The Needleman-Wunsch algorithm was used to create a 2D matrix D using the global alignment function. Sequence A, Sequence B, the substitution matrix, and the gap penalty \newline Example: Sequence A="GATA"
Sequence="CTAC” with a match score of 1, mismatch score of -1, and gap score of -2.
\newline The D matrix is generated as below:
\newline[[ 0 -2 -4 -6 -8]
 \newline[-2 -1 -3 -5 -5]
 \newline[-4 -3 -2 -2 -4]
 \newline[-6 -5 -2 -3 -3]
 \newline[-8 -7 -4 -1 -3]]
 \newline As we traverse the input sequence “GATA”
and “CTAC” we get the global alignments [("GATA-","C-TAC"),("GATA-","-CTAC")]
\newline
\newline Example: Sequence A="ACTAG" Sequence B="AGCT” with a match score
of 1, mismatch score of -1, and gap score of -2.
\newline The D matrix is generated as below:
\newline[[  0  -2  -4  -6  -8]
\newline[ -2   1  -1  -3  -5]
\newline[ -4  -1   0   0  -2]
\newline[ -6  -3  -2   -1  1]
\newline[ -8  -5  -4   -3  -1]
\newline[ -10  -7  -4  -5   -3]]
\newline we get the global alignments such as [("A-CTAG", "AGCT--")]



\section{Local Alignment}
The Smith-Waterman algorithm was used to create a 2D matrix D with the help of a function called local alignment, which takes variables. Sequence A, Sequence B, the substitution matrix, and the gap penalty
\newline
\newline Example: Sequence A="ACTG"
Sequence B="GTCA” with a match score
of 1, mismatch score of -1, and gap score of -2.
\newline The D matrix is generated as below:
\newline [0 0 0 0 0]
\newline [0 0 0 0 1]
\newline [0 0 0 1 0]
\newline [0 0 1 0 0]
\newline [0 1 0 0 0]
\newline Traversing for the input sequence “ACTG” and “GTCA” we get the global alignments such as [("A", "A")]\newline 
\newline Example: Sequence A="GTACT" Sequence B="CTAGA” with a match score
of 3, mismatch score of -1, and gap score of -2.
\newline The D matrix is generated as below:
\newline [0 0 0 0 0 0]
\newline [0 0 0 0 1 0]
\newline [0 0 1 0 0 0]
\newline [0 0 0 2 0 1]
\newline [0 1 0 0 1 0]
\newline [0 0 2 0 0 0]
\newline
As we traversing D,for the input sequence "GTACT" and "CTAGA” we get the global alignments such as [("TA", "TA")]
 
\section{Custom Alignment}
Taking first name and last name in lowercase and concatenating them, firstname="venkatankit" ,lastname="gundala"
\newline when we concatenate string is "venkatankitgundala" ,\newline
With the help of function named local alignment from above problem which takes variables,Sequence A,Sequence B,substitution matrix,gap penalty\newline
match=2,semi-match=+1,mismatch=-1,gap=-2
Sequence A="venkatankitgundala" Sequence B="hequickbrownfoxjumpsoverthelazydog”\newline
\newline The score matrix is included in file 1002069069\textunderscore S.txt \newline
\newline The pretty print is included in 1002069069\textunderscore D.txt \newline
\newline The local alignment for the above strings is [("ve-nkata", "verthela"), ("venkatan", "verthela"), ("ve-nkatankitg","verthelazydog"), ("tgundala", "verthela")]

\section{Difficulty Adjustment}
It took me around 20 hours to complete this assignment and I felt that the custom alignment part was a bit tough to complete

\end{document}
